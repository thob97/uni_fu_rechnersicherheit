\input{src/header}
\graphicspath{ {./src/} } 
\usepackage{hyperref}

\newcommand{\dozent}{Volker Roth}
\newcommand{\tutor}{Oliver Wiese}
\newcommand{\tutoriumNo}{02\\Materialien: Latex, VSC, Skript}
\newcommand{\ubungNo}{06}
\newcommand{\veranstaltung}{Rechnersicherheit}
\newcommand{\semester}{SoSe 21}

% /////////////////////// BEGIN DOKUMENT /////////////////////////
\begin{document}
\input{src/titlepage}

% /////////////////////// Task 1 /////////////////////////
\section{Password Generation using Context Free Grammars}
\begin{itemize}

    % /////////////////////// a /////////////////////////
    \item {\itshape Consider the following probabilistic context-free grammar.}
    
    {\Large
    \begin{tabular}{c|c}
        \textbf{Production rule} & \textbf{Probability}\\ 
        \hline  
        S \xrightarrow{} D_1L_3S_2D_1 & 0.75 \\
        S \xrightarrow{} L_3D_1S_1 & 0.25 \\
        \hline  
        D_1 \xrightarrow{} 4 & 0.60 \\
        D_1 \xrightarrow{} 5 & 0.20 \\
        D_1 \xrightarrow{} 6 & 0.20 \\
        \hline  
        S_1 \xrightarrow{} ! & 0.65 \\
        S_1 \xrightarrow{} \% & 0.30 \\
        S_1 \xrightarrow{} \# & 0.05 \\
        \hline 
        S_2 \xrightarrow{} \$\$ & 0.70 \\
        S_2 \xrightarrow{} ** & 0.30 \\
    \end{tabular}
    }
    
    \item {\itshape and the following priority queue:}
    
    {\Large
    \begin{tabular}{c|c|c|c}
        \textbf{Base Struct} & \textbf{Pre-Terminal} & \textbf{Probability} & \textbf{Pivot Value}\\ 
        \hline  
        \textcolor{blue}{D}_1^{\textcolor{red}{0}} L_3 S_2^{\textcolor{red}{0}} D_1^{\textcolor{red}{0}} & 4L_3\$\$4 & 0.188 & \textcolor{blue}{0} \\
        
        L_3\textcolor{blue}{D}_1^{\textcolor{red}{0}} S_1^{\textcolor{red}{0}} & L_34! & 0.097 & \textcolor{blue}{0} \\
    \end{tabular}
    }  
    
\end{itemize} 
   
\newpage
\begin{enumerate}[(a)]
    \item {\itshape Calculate the next five pre-terminal structures that the enumerator (as discussed in class)does output and the resulting priority queue.}
    \begin{enumerate}[1.]
        \item After the first password was extracted:
        
        {\Large
        \begin{tabular}{c|c|c|c}
            \textbf{Base Struct} & \textbf{Pre-Terminal} & \textbf{Probability} & \textbf{Pivot Value}\\ 
            \hline  
            
            L_3\textcolor{blue}{D}_1^{\textcolor{red}{0}} S_1^{\textcolor{red}{0}} & L_34! & 0.097 & \textcolor{blue}{0} \\
            
            D_1^{\textcolor{red}{0}} L_3 \textcolor{blue}{S}_2^{\textcolor{red}{1}} D_1^{\textcolor{red}{0}} & 4L_3**4 & 0.081 & \textcolor{blue}{1} \\
            
            \textcolor{blue}{D}_1^{\textcolor{red}{1}} L_3 S_2^{\textcolor{red}{0}} D_1^{\textcolor{red}{0}} & 5L_3\$\$4 & 0.063 & \textcolor{blue}{0} \\
            
            D_1^{\textcolor{red}{0}} L_3 S_2^{\textcolor{red}{0}} \textcolor{blue}{D}_1^{\textcolor{red}{1}} & 4L_3\$\$5 & 0.063 & \textcolor{blue}{2} \\
        \end{tabular}
        }  
        
        \item After the second password was extracted:
        
        {\Large
        \begin{tabular}{c|c|c|c}
            \textbf{Base Struct} & \textbf{Pre-Terminal} & \textbf{Probability} & \textbf{Pivot Value}\\ 
            \hline  
            
            D_1^{\textcolor{red}{0}} L_3 \textcolor{blue}{S}_2^{\textcolor{red}{1}} D_1^{\textcolor{red}{0}} & 4L_3**4 & 0.081 & \textcolor{blue}{1} \\
            
            \textcolor{blue}{D}_1^{\textcolor{red}{1}} L_3 S_2^{\textcolor{red}{0}} D_1^{\textcolor{red}{0}} & 5L_3\$\$4 & 0.063 & \textcolor{blue}{0} \\
            
            D_1^{\textcolor{red}{0}} L_3 S_2^{\textcolor{red}{0}} \textcolor{blue}{D}_1^{\textcolor{red}{1}} & 4L_3\$\$5 & 0.063 & \textcolor{blue}{2} \\
            
            L_3 D_1^{\textcolor{red}{0}} \textcolor{blue}{S}_1^{\textcolor{red}{1}} & L_34\% & 0.045 & \textcolor{blue}{1} \\
            
            L_3\textcolor{blue}{D}_1^{\textcolor{red}{1}} S_1^{\textcolor{red}{0}} & L_35! & 0.0325 & \textcolor{blue}{0} \\
        \end{tabular}
        }  
        
    \end{enumerate}
    
    
\end{enumerate}


\newpage
% /////////////////////// Task 2 /////////////////////////
\section{Code Review}
{\itshape Review and test the Python code of your peers. Your focus should be the implementation of the password-based authentication and consider at least the following aspects:}
\begin{enumerate}[1.]
    % /////////////////////// Peer 1 /////////////////////////
    \item \textbf{Peer}:
    \begin{enumerate}[(a)]
        % /////////////////////// a /////////////////////////
        \item {\itshape Used hash function and configuration.}
        
        \textbf{Implemented}
        \lstinputlisting[language=python, linerange={143-145}, firstnumber = 143]{src/u6/user_manager.py}
        
        % /////////////////////// b /////////////////////////
        \item {\itshape Usage and generation of salts and randomness.}
        
        \textbf{Implemented}
        \lstinputlisting[language=python, linerange={188-197}, firstnumber=188]{src/u6/user_manager.py}

        % /////////////////////// c /////////////////////////
        \item {\itshape Duplicate user names.}
        
        \textbf{Prevented}
        \lstinputlisting[language=python, linerange={41-48, 52-57}, firstnumber=41]{src/u6/user_manager.py}
\newpage    
        % /////////////////////// d /////////////////////////
        \item {\itshape Simultaneous creation of users with the same name.}
        
        \textbf{Not prevented. Still possible.} I added a sleep in the server, just to demonstrate it, even without the sleep it can happen, as we work with threads. To fix this you can use locks in the server authentication method.
        \lstinputlisting[language=python, linerange={41-57}, firstnumber=41]{src/u6/user_manager.py}
        \includegraphics[width=\linewidth]{src/u6/nolock.png}
\newpage
        % /////////////////////// extra /////////////////////////
        \item {\itshape \textbf{Extra:} I was able to find some security risks in your code with bandit.}
        \lstinputlisting[language=python, linerange={9-44}, firstnumber=9]{src/u6/bandit_output_peer_1.txt}
        \begin{itemize}
            \item To fix the first two to security risks you could use the bcrypt module which has a 'safer' hashing function. Here is an example syntax:
            \lstinputlisting[language=python, linerange={1-3}, firstnumber=1]{src/u6/example_1.txt}
            
        \item To fix the third/fourth security risk you would need to create 'real' randomness, which can be done with a quantum PCs, but as you most likely don't have one right now, you can also use the bcrypt module to create a 'safer' salt. With this salt generator bandit will stop throwing an error. Here is the example:
        \lstinputlisting[language=python, linerange={5}, firstnumber=5]{src/u6/example_1.txt}
        \end{itemize}
        
    \end{enumerate}
    
    
    
\newpage
    % /////////////////////// Peer 2 /////////////////////////
    \item \textbf{Peer}:
    \begin{enumerate}[(a)]
        % /////////////////////// a /////////////////////////
        \item {\itshape Used hash function and configuration.}
        
        \textbf{Implemented}
        \lstinputlisting[language=python, linerange={57-57}, firstnumber = 57]{src/u6/server.py}
        
        % /////////////////////// b /////////////////////////
        \item {\itshape Usage and generation of salts and randomness.}
        
        \textbf{Somewhat implemented.} Salt is used but not random!
        \lstinputlisting[language=python, linerange={11-11}, firstnumber=11]{src/u6/server.py}

        % /////////////////////// c /////////////////////////
        \item {\itshape Duplicate user names.}
        
        \textbf{Prevented}
        \lstinputlisting[language=python, linerange={64-71}, firstnumber=64]{src/u6/server.py}
        
        % /////////////////////// d /////////////////////////
        \item {\itshape Simultaneous creation of users with the same name.}
        
        \textbf{Somewhat prevented. Might still be possible.} But I could not test it, because the server does not allow it when multiple clients try to register / login at the same time. This seems more like a bug than a feature. If this would be fixed, simultaneous creation of users with the same name would be possible. To fix this you can use locks.
        Below is a example on how I tried to register with two clients at the same time. I added a print in the client to show the 'error'.
        
        \includegraphics[width=\linewidth]{src/u6/error2.png}
        
        Here is the code snippet of you client. I added the lines 79 and 80.
        \lstinputlisting[language=python, linerange={66-80}, firstnumber=66]{src/u6/client.py}

\newpage
        % /////////////////////// extra /////////////////////////
        \item {\itshape \textbf{Extra:} Bandit found no further security risky.}
        
        % /////////////////////// Error /////////////////////////
        \item {\itshape \textbf{Errors:} I had some errors running your Dockerfile.}
        \begin{itemize}
            \item Your Dockerfile searched for 'client/client.py' and 'server/server.py', but they are named 'server\textbackslash server.py' and 'server\textbackslash server.py'. I fixed it by just renaming them. Here is the log file of the first time I ran your Dockerfile:
            \lstinputlisting[language=python]{src/u6/error.txt}

        \end{itemize}

    \end{enumerate}
    
\end{enumerate}

\end{document}