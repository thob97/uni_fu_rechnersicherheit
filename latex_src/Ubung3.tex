\input{src/header}
\graphicspath{ {./src/} } 
\usepackage{hyperref}

\newcommand{\dozent}{Volker Roth}
\newcommand{\tutor}{Oliver Wiese}
\newcommand{\tutoriumNo}{02\\Materialien: Latex, VSC, Skript}
\newcommand{\ubungNo}{03}
\newcommand{\veranstaltung}{Rechnersicherheit}
\newcommand{\semester}{SoSe 21}

% /////////////////////// BEGIN DOKUMENT /////////////////////////
\begin{document}
\input{src/titlepage}

% /////////////////////// Task 1 /////////////////////////
\section{Password models}
\begin{enumerate}[(a)]
    % /////////////////////// a /////////////////////////
    \item {\itshape Choose your tool to analyze the password set. We recommend python (with pandas)or R but you can choose an other tool. Be aware that most CSV-Tools can not handle a huge file.}
    \begin{itemize}
        \item Wir haben das pandas CSV-Tool benutzt. Da die zu analysierende Datei jedoch keinen seperator zwischen Passworthäufigkeit und Passwort hat, haben wir folgendes versucht:
        \begin{enumerate}[1.]
            \item At first we used a white space as separator, however, this caused some passwords to be left out or incorrectly recorded, which threw several errors as consequence.
            \lstinputlisting[language=python]{src/u3/a1.py}	
            
            \item After that we used a regex as separator, however, the number of errors only decreased slightly. It also took a lot longer to analyze the file. 
            \lstinputlisting[language=python]{src/u3/a2.py}
            Snipped of passwords containing errors / not loading with pandas (The passwords in this case are whitespaces):
            \lstinputlisting[]{src/u3/a2_errors.txt}
\newpage            
            \item Then we came up with the idea of using preprocessing before reading the file with pandas. There should now be a comma between each password and the password frequency. The runtime has been improved, but there are several 'password errors' again, because many passwords contain commas.
            \lstinputlisting[,linerange={1-20}]{src/u3/a3_errors.txt}
 
            \item In preprocessing we replaced the comma with a tab. Now there are almost no password errors anymore.
            \lstinputlisting[language=python]{src/u3/a4.py}									
            \lstinputlisting[]{src/u3/a4_errors.txt}

        \end{enumerate}
    \end{itemize}
\newpage        
    % /////////////////////// b /////////////////////////
    \item {\itshape Describe the password set and passwords, e.g. size, distribution, password length, errors in the set. Each description should include at least the center (mean, median or mode), the dispersion (variance, range, percentile).}
    \begin{itemize}
        \item \lstinputlisting[language=python,
            linerange={31-56}]
            {src/u3/password.py}
            \lstinputlisting[]
            {src/u3/stats_file.txt}
    \end{itemize}

    % /////////////////////// c /////////////////////////
    \item {\itshape Take a look at the ten most frequent passwords. Give a password (creation) policy for them. Your policy can be informal or very formal.}
    \begin{itemize}
        \item The ten most frequent passwords are the following:
        \lstinputlisting[]{src/u3/10_most_frequent.txt}
        \item These passwords underline the following policy's:
        \begin{enumerate}[1.]
            \item 5-9 characters
            \item characters can be from the alphabet [0-9] or [a,b-z] 
        \end{enumerate}
    \end{itemize}
    
        
    
    
    % /////////////////////// d /////////////////////////
    \item {\itshape There is no single password policy. Select all passwords which are 7 to 32 characters long, contain at least one digit and at least one upper case letter. We call this set P1.}
    \begin{itemize}
        \item \lstinputlisting[language=python,
        linerange={63-74}]
        {src/u3/password.py}
    \end{itemize}	
    
    % /////////////////////// e /////////////////////////
    \item {\itshape P1 will  no  contain  all  passwords  which  match  to  our  policy.  How  do  we  handle passwords which do not appear but match to our policy? You either extend P1 or not. Give the reasons for your decision.}
    \begin{itemize}
        \item We did not extend P1. The strength of a password policy is measured empirically with a large sample of passwords. We want to know how strong the password policy is in a real world scenario. In a real world scenario, the passwords are created by people. Therefore the provided password sample file is ideal, as it contains passwords of accounts from real people. This means that if we were to extend P1 with passwords generated by an algorithm, it would falsify our desired result, "because passwords are not usually distributed uniformly" (quote from Lecture 3)
    \end{itemize}
    
    
    % /////////////////////// f /////////////////////////
    \item {\itshape Calculate the probability of each password in P1.}
    \begin{itemize}
        \item \lstinputlisting[language=python,
            linerange={76-79}]
            {src/u3/password.py}
            \lstinputlisting[
            linerange={0-10}]
            {src/u3/p1.txt}
    \end{itemize}
    
    % /////////////////////// g /////////////////////////
    \item {\itshape Calculate Shannon Entropy of P1.}
    \begin{itemize}
    \item We calculated the Shannon Entropy with the following formula from the lecture:
        \\{\Large $H(X) = - \sum_{i=1}^\N Pr[X = x_i] log_2 (Pr[X = x_i]) $}
        \lstinputlisting[language=python,
            linerange={81-84}]
            {src/u3/password.py}
    
    The result is 19.10637335027745
    \end{itemize}   
    
    % /////////////////////// h /////////////////////////
    \item {\itshape Assume  an  online  adversary  with  unlimited  guesses  likes  to  break  200  accounts. Estimate the expected number of guesses.}
    \begin{itemize}
        \item For calculating the expected number of guesses we also used a formula from the lecture:
            \\{\Large $G(X) = \sum_{i=1}^\N i \cdot Pr[X = x_i] $}
            \lstinputlisting[language=python,
                linerange={86-94}]
                {src/u3/password.py}
            The result is 57652515
    \end{itemize}
    
    % /////////////////////// i /////////////////////////
    \item {\itshape Assume  an  online  adversary  likes  to  break  at  least  50  of  200  accounts.  Estimate adversary’s work.}
    \begin{itemize}
    \item We used the following formula from the lecture:
        \\{\Large $ \sum_{i=1}^ \beta Pr[X = x_i] \ge \frac {l} {k} $}
        \lstinputlisting[language=python,
            linerange={109-117}]
            {src/u3/password.py}
        The result is 62693
    \end{itemize}
    
    
    %This takes 7 minutes on my computer.
    %\begin{lstlisting}
    %#assume sorted
    %def beta_succ(beta,p1):
    %    return p1['probability'][0:beta].sum()

    %#this is slooow
    %def i(p1):
    %    alpha = 50/200
    %    a_work = min([beta for beta in list(range(len(p1))) if beta_succ(beta,p1)>=alpha  ])
    %    except_fail = (1-beta_succ(a_work,p1))*a_work
    %    except_succ = (p1['probability'][0:a_work]*list(range(1,a_work+1))).sum()
    %    return except_fail+except_succ
    %\end{lstlisting}
    %Result is 52073
    
    % /////////////////////// j /////////////////////////
    \item {\itshape Formalize your own password policy and repeat the above estimations.}
    \begin{itemize}
        \item Our passowrd policy's follows the following rules:
            \begin{enumerate}[1.]
                \item 8-64 characters
                \item contains at least two digit
                \item contains at least two upper case letter
                \item contains at least one lower case letter
            \end{enumerate}
        \item Result for the above estimations:
                \begin{enumerate}[a)]
                    \setcounter{enumii}{5}
                    \item \textbf{Probability}
                        \lstinputlisting[language=plain,
                            basicstyle=\scriptsize,
                            linerange={0-10}]
                            {src/u3/task_j_file.txt}
                    \item \textbf{Shanon Entropy} result is 16.036519282236434
                    \item The adversary needs 6565494.616061121 guesses to \textbf{break 200 accounts}
                    \item The adversary needs 14473 guesses to \textbf{break 50 of 200 accounts}
                \end{enumerate}
    \end{itemize}

    
    
    % /////////////////////// k /////////////////////////
    \item {\itshape Compare the password policies and estimations.}
    \begin{enumerate}[]
        \item The main difference between p1 and our own password policy is that we require two capital letters and and two numbers. We thought that this would make the passwords created by users more secure. However, the Shannon entropy and the number of guesses for cracking accounts have decreased. Perhaps this is due to the lower password sample size, or it is because our password policy restricts the number of possible passwords and thus becomes weaker. However, we tend to assume the first. \textcolor{red}{The passwords in P2 would likely be a lot stronger against a blind brute-force attack. However since the probabilities we used during the analysis don’t assume a blind attack, but rather that the attacker knows P2 in advance, the passwords are a lot weaker in this analysis. This is because there are fewer of them compared to P1 and thus trying them all is a lot easier than trying all of the passwords in P1.}
        
        \begin{tabular}{l|l|l}
            \textbf{Estimations} & \textbf{P1} & \textbf{Own password policy}\\ 
            \hline  
            Password sample size & 858.415 & 72.772 \\
            Shanon Entropy & 19 & 16\\
            Guesses to break 200 accounts & 57.652.515 & 6.565.494 \\
            Guesses to break 50 of 200 accounts & 62.693 & 14.473\\
        \end{tabular}
        
    \end{enumerate} 
    
\end{enumerate}

% /////////////////////// Task 2 /////////////////////////
\section{Project}
\begin{enumerate}[(a)]
    % /////////////////////// a /////////////////////////
    \item \itshape{Setup Docker on your machine (or on a virtual machine).}
    
    % /////////////////////// b /////////////////////////
    \item \itshape{Add a docker image for your project. You find an example in the resources folder.}
    \begin{enumerate}
        \item Some use full docker commands:
        \lstinputlisting[language=python]{src/u3/docker_commands.txt}
        \item Docker and docker-compose files:
        \lstinputlisting[language=python]{src/u3/Dockerfile1}
        \lstinputlisting[language=python]{src/u3/Dockerfile1}
        \newpage
        \lstinputlisting[language=python,basicstyle=\scriptsize,]{src/u3/docker-compose.yml}
    \end{enumerate}
    % /////////////////////// c /////////////////////////
    \item \itshape{Before using third party dependencies. We might should talk about them and potential threats. Read (or watch the video) the paper Small World with High Risks: A Study of Security Threats in the npm Ecosystem and list at least 3 risks using npm.}
    \begin{enumerate}[1.]
        \item \textcolor{red}{Using npm packages could cause security vulnerabilities because of malicious packages  which  were  installed  manually  or  by  the  package  manager  due  to dependencies of other packages.}
        \item \textcolor{red}{lot of popular, widely used packages are potential targets for injecting malwa-re, so that software or other libraries which use the packages are compromisedtoo (without knowing).}
        \item \textcolor{red}{Some maintainers have a lot of packages and are referenced in thousands of otherpackages, so that if anyone compromises one of these accounts, the consequencesare enormous.}
        \item \textcolor{red}{Due  to  no  consistent  vetting  process,  just  a  few  security  vulnerabilities  canbe  discovered  by  some  individuals,  research  work  or  in  worst  case  in  resultof an attack. So there are about 1600 detected and more undetected securityvulnerabilities in packages, which are often used.}
    \end{enumerate}
    % /////////////////////// d /////////////////////////
    \item \itshape{Is this relevant for Python and our project? Give a short statement. You might find some research about Python, too.}
    \begin{itemize}
        \textcolor{red}{\textbf{Solution:} Python has, as well as javascript an ecosystem like npm, which is based on open access developement and is called pip. Long story short, the main issues about malicious packages, user accounts which are compromised to manipulate widely used packages and even the execution of malicious code in the background after importing packages, which enables the attacker to control camera or microphone of the target system remotely, are present in the pip ecosystem of python as well.For additional information see:https://arxiv.org/pdf/2102.06301.pdf In conclusion, that means that it is relevant for Python and our project.}
    \end{itemize}
\end{enumerate}






\end{document}

