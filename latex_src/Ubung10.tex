\input{src/header}
\graphicspath{ {./src/} } 
\usepackage{hyperref}

\newcommand{\dozent}{Volker Roth}
\newcommand{\tutor}{Oliver Wiese}
\newcommand{\tutoriumNo}{02\\Materialien: Latex, VSC, Skript}
\newcommand{\ubungNo}{03}
\newcommand{\veranstaltung}{Rechnersicherheit}
\newcommand{\semester}{SoSe 21}

% /////////////////////// BEGIN DOKUMENT /////////////////////////
\begin{document}
\input{src/titlepage}

% /////////////////////// Task 1 /////////////////////////
\section{SA and EM}
In class we discussed static analysis and execution monitors. We like to recap and discuss some concepts.
\begin{enumerate}[(a)]
    % /////////////////////// a /////////////////////////
    \item {\itshape Discuss differences between safety and liveness properties.}
    \begin{itemize}
        \item Safety stipulate that “bad things never happen” during execution.
        \item in contrast to liveness properties which stipulate that, eventually, “good things will happen” during execution.
    \end{itemize}
     
    % /////////////////////// b /////////////////////////
    \item {\itshape Discuss differences between SA and EM mechanisms.}
    \begin{itemize}
        \item A SA mechanism takes a program $a$ and {decides if it fits the policy, afterwards it returns a program}. If it does, the initial program $a$ is returned. Else a different program is returned, which halts on every input.
        \item A EM mechanism takes a program $a$ and returns a created program $b$. $b$ runs $a$ and \\ {observes in the run time if $a$ violates the policy}. If it does, $b$ halts. {$b$ can only observe a single execution of $a$.}
        \item So SA happens befor runtime and EM in runtime
        \item Also Every SA mechanism is also enforceable by an EM mechanism, but not the contrary.
        
    \end{itemize}

    % /////////////////////// c /////////////////////////
    \item {\itshape Give a practical example of SA mechanisms.}
    \begin{itemize}
        \item type-safe for program languages (such as Java)
        \item standard virus scanners
    \end{itemize}


    % /////////////////////// d /////////////////////////
    \item {\itshape Give a practical example of EM mechanisms.}
    \begin{itemize}
        \item Software testing (memory leaks, out-of-bounds array accesses, race conditions, atomicity, etc.)
        \item Auditing and Logging
    \end{itemize}

    
\end{enumerate}

% /////////////////////// Task 2 /////////////////////////
\section{Security Policies in our Project}
In class we discussed security policies in a very formal way. In this exercise we focus on some  practical  aspects  of  security  policies  in  context  of  your  project  (chat server and client).
\begin{enumerate}[(a)]
    % /////////////////////// a /////////////////////////
    \item {\itshape Give  three  relevant  security  policies  for  your  project.  The  description  of  security policies can be informal.}
    \begin{enumerate}[1.]
        \item Users should not be able to see whenever a new user connected or disconnected from the server
        \item User should not be able to change any data in the programm (code)
        \item Policies for different account privileges. (Admins can delete msgs, while users are unable to)
    \end{enumerate}
    
    % /////////////////////// b /////////////////////////
    \item {\itshape For each above security policy: Who should enforce the policy? The operating system, your program or someone else?}
    \begin{enumerate}[1.]
        \item The 1st and 3rd policies are somewhat similar; both could also be \underline{implemented in code}. For 1. simply omit the option, and for 3. restrict the option to admin accounts.
        \item 2. Should be \underline{handled by the operating system}. 
    \end{enumerate}

    % /////////////////////// c /////////////////////////
    \item {\itshape For each security policy: How can be your security policies enforced? With a SA or EM mechanism?}
    \begin{itemize}
        \item We'd choose for every policy a EM mechanims. (Especially because an em mechanism can simulate an sa mechanism :)
    \end{itemize}

\end{enumerate}

\end{document}